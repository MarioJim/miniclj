
\documentclass[11pt]{scrreprt}
\usepackage[utf8]{inputenc}

\usepackage[T1]{fontenc}
\usepackage{helvet}
\renewcommand{\familydefault}{\sfdefault}
\addtokomafont{chapter}{\LARGE}
\renewcommand*{\chapterformat}{\thechapter\hspace{0.4cm}}
\RedeclareSectionCommand[style=section]{chapter}

\usepackage{graphicx}
\usepackage{sectsty}
\usepackage{titlesec}

\usepackage{xcolor}
\definecolor{LightGray}{gray}{0.9}
\usepackage{minted}
\usemintedstyle{borland}
\setminted{bgcolor=LightGray,fontsize=\small}

% Margins
\topmargin=-0.45in
\evensidemargin=0in
\oddsidemargin=0in
\textwidth=6.5in
\textheight=9.0in
\headsep=0.25in


\begin{document}

{\centering
{\huge
miniclj\par}
{\LARGE
User Manual\par}}

\tableofcontents

\chapter{About the language}
This project's aim is to create a compiler and virtual machine for a lisp-based language with similar semantics to Clojure. The base functions and data structures will be supported, and they must be accessible either through a Command-Line Interface or inside a web context.

\section{Differences and limitations compared to Clojure}
Other than not including a broader standard library compared to Clojure, miniclj has some differences and limitations, like:
\begin{itemize}
    \item Support for symbols during runtime isn't supported because they must be linked to a memory address during compilation
    \item Expressions and lists are evaluated eagerly, miniclj doesn't support lazy sequences
    \item Lambda functions don't capture their enclosing environment/scope
    \item Support for macros wasn't implemented
    \item Code is strictly single threaded, and there is no support for using concurrency controls like atoms or promises
\end{itemize}

\chapter{Data types}
\section{Numbers}

\section{Strings}

\section{Lists}

\section{Vectors}

\section{Maps}

\section{Sets}

\section{Nil}


\chapter{Callables}
\section{Collection functions}
\subsection{Access}
\subsubsection{\texttt{first}}
\begin{minted}{clojure}
(first collection)
\end{minted}

Returns the first item in the collection. If the collect


\subsection{Creation}

\subsection{Generation}

\subsection{Modification}

\subsection{Transducers}


\section{Comparison operations}

\section{Conditionals}

\section{Cycles}

\section{Factor operations}

\section{Grouping functions}

\section{I/O functions}

\section{Lambda functions}

\section{Scope functions}

\section{Typecasting functions}


\end{document}
